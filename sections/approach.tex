The methodological approach consists of the following steps:
\begin{enumerate}
    \item Existing methods for training a neural network will be identified and explained. Goal is to give a brief introduction in this area so that non-expert reader can follow the rest of the thesis. These methods will be used to train a deep neural network for age estimation of a person in the image.
    
    \item Research about the different state of the art methods for generating \textit{adversarial examples}, i.e. images which are not correctly classified by DNN needs to be conducted. Focus here is on the approaches that have a possibility of targeted misclassification. While treating a DNN from the first step as a white-box, those methods will be used to construct adversarial examples. Results of the attacks will be presented and analyzed.
    
    \item Research about the targeted black-box attack methods will be done and those methods will explained and used to generate an adversarial inputs for a DNN  from the first step, but this time while treating it as a black box. Results of different algorithms will be compared. 
    
    \item As the last step, combining the domain knowledge and an existing attack method, a new algorithm will be implemented and images for semi-targeted misclassification will be constructed. Results will be compared against targeted black-box approaches. The way it will be compared is the following: a target label for a targeted version of the attack will be the median value of the set of labels in a semi-targeted version of the attack. If a DNN outputs any label out of the set of the labels, an attack will be considered as a successful no matter if it is a targeted or a semi-targeted misclassification attack.
\end{enumerate}