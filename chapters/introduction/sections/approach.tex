The methodological approach consists of the following steps:
\begin{enumerate}
    \item I write about the background knowledge needed for training a neural network. The goal is to give a brief introduction to this area so that a non-expert reader can follow the rest of the thesis. Using that knowledge, I train a deep neural network for age estimation of a person in the image.
    
    \item I conduct a research about different state of the art methods for generating \textit{adversarial examples}, i.e. images which are not correctly classified by DNN. The focus here is on the approaches that address targeted misclassification. While treating a DNN from the first step as a white-box, I use those methods to construct adversarial examples. Afterwards, I present and analyze the results of the attacks.
    
    \item I make a literature survey on targeted black-box attack methods, explain those methods and use them to generate adversarial inputs for the DNN from the first step, but this time while treating it as a black box. I compare the results of different algorithms.
    
    \item As the last step, combining the domain knowledge and an existing attack method, I implement a new adversarial algorithm and using that algorithm, I construct images for semi-targeted misclassification. I compare results against targeted black-box approaches.
\end{enumerate}
