The methodological approach consists of the following steps:
\begin{enumerate}
    \item Existing methods for training a neural network are identified and explained. The goal is to give a brief introduction to this area so that a non-expert reader can follow the rest of the thesis. These methods are used to train a deep neural network for age estimation of a person in the image.
    
    \item Research about different state of the art methods for generating \textit{adversarial examples}, i.e. images which are not correctly classified by DNN is conducted. The focus here is on the approaches that have a possibility of targeted misclassification. While treating a DNN from the first step as a white-box, those methods are used to construct adversarial examples. Results of the attacks are presented and analyzed.
    
    \item A literature survey on the targeted black-box attack methods is done and those methods are explained and used to generate adversarial inputs for a DNN  from the first step, but this time while treating it as a black box. Results of different algorithms are compared. 
    
    \item As the last step, combining the domain knowledge and an existing attack method, a new algorithm is implemented and images for semi-targeted misclassification are constructed. The results are compared against targeted black-box approaches. The way it is compared is the following: a target label for a targeted version of the attack is the median value of the set of labels in a semi-targeted version of the attack. If a DNN outputs any label out of the set of the labels, an attack is  considered as a successful no matter if it is a targeted or a semi-targeted misclassification attack.
\end{enumerate}
