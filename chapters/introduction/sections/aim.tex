To the best of my knowledge, nobody executed an attack against a DNN in the domain of age estimation yet. Hence, one of the first questions which this thesis aims to answer is the question whether already existing techniques can be used for attacks in this domain.

In order to further define aim of this work, an explanation of the major difference in between two types of attacks against a neural network is necessary, namely the difference in between \textit{white-box} attack and \textit{black-box} attack. 
In white-box attacks, an adversary has all the information about the DNN under attack. The internal structure of the neural network, all the implementation details and values of all the variables in any moment are known to the attacker. In other words, the attacker has access to the source code and nothing is hidden.
On the other hand, in black-box attacks an adversary doesn't have access to all the information as in white-box. Depending on the precise definition of the black-box, more or less information is provided. In this thesis, the only capability of the black-box adversary is to observe a label given by the DNN to chosen inputs.

A natural question is - which algorithm is the best one for the white-box attack and which one for the black-box attack? Several algorithms will be run in the different settings and results will be compared. That will answer a question which algorithm to use in which scenario.

Furthermore, in the domain of age estimation, it makes sense to be less strict about the targeted label. For instance, if an image of a minor is classified as an image of a person over a fifty years old, this could lead to the same consequences, no matter if it is classified as a 55 or 65 years old person. More precisely, if the goal is to hit any label from a specific group of labels, I define that attack as \textit{semi-targeted misclassification}. Of course, a group of labels must be smaller than a whole set of possible labels, otherwise the task is trivial. In general, this attack makes sense in any environment where labels can somehow be ordered.

One already existing black-box approach will be adapted to this more relaxed setting. Such a relaxation is not yet introduced in the literature since it's very domain specific so results can't be compared against a previous work, but results can be and will be compared with targeted black-box misclassification attacks. This comparison is explained in more details in Section \ref{approach}. 

In terms of development, a DNN will be implemented which will receive an image as an input and output will be how old the person in the image is. In other words, a DNN for age estimation will be trained. All the attacks will be executed against this DNN. 

Next, the framework will be constructed for a white-box and a black-box targeted misclassification attack. In other words, while treating the DNN as a white-box or a black-box, images will be constructed in a way that the targeted DNN outputs a specific year. Finally, the framework will be expanded by the ability to craft semi-targeted black-box misclassification attacks.