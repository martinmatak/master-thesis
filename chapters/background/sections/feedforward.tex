\textit{Feedforward neural networks} or \textit{multilayer peceptrons} (MLPs) are the essential deep learning models. A feedforward neural network defines a mapping $f (\pmb{x} ; \pmb{\theta}) = \pmb{y}$ and learns the value of the parameters $\pmb{\theta}$ that result in the best function approximation.

One example of such a model is shown in the Figure \ref{fig:basic-cnn}. A neural network consists of several \textit{layers}. There is always one \textit{input} layer, followed by one or more \textit{hidden} layers and finally, there is an \textit{output layer}. The number of layers (without an input layer) defines a \textit{depth} of the model. There is no precise definition, but neural networks with more than three layers are called \textit{deep} neural networks. 

When it comes to feedforward neural networks, a network is an directed acyclic graph. Vertices of such a graph are called \textit{units} or \textit{neurons} and edges are called \textit{weights}. Vertices represent scalar functions of an input. Weights are the parameters $\pmb \theta$ which a neural network learns. Edges define data flow.

The \textit{Convolutional Neural Networks} (CNNs) which are described in the Subsection \ref{subsection:convolutionalNN} are a specific kind of feedforward neural networks. If there are cycles in a graph, we are probably talking about the \textit{Recurrent Neural Networks} (RNNs)which are used in domains where a context is important. For instance in the Natural Language Processing. However, RNNs are out of the scope of this thesis and won't be further analysed.

\begin{figure}
\begin{tikzpicture}[
plain/.style={
  draw=none,
  fill=none,
  },
net/.style={
  matrix of nodes,
  nodes={
    draw,
    circle,
    inner sep=10pt
    },
  nodes in empty cells,
  column sep=2cm,
  row sep=-9pt
  },
>=latex
]
\matrix[net] (mat)
{
|[plain]| \parbox{1.3cm}{\centering Input\\layer} & |[plain]| \parbox{1.3cm}{\centering Hidden\\layer} & |[plain]| \parbox{1.3cm}{\centering Output\\layer} \\
& |[plain]| \\
|[plain]| & \\
& |[plain]| \\
  |[plain]| & |[plain]| \\
& & \\
  |[plain]| & |[plain]| \\
& |[plain]| \\
  |[plain]| & \\
& |[plain]| \\    };
\foreach \ai [count=\mi ]in {2,4,...,10}
  \draw[<-] (mat-\ai-1) -- node[above] {Input \mi} +(-2cm,0);
\foreach \ai in {2,4,...,10}
{\foreach \aii in {3,6,9}
  \draw[->] (mat-\ai-1) -- (mat-\aii-2);
}
\foreach \ai in {3,6,9}
  \draw[->] (mat-\ai-2) -- (mat-6-3);
\draw[->] (mat-6-3) -- node[above] {Ouput} +(2cm,0);
\end{tikzpicture}

\caption{An example of the feedforward neural network architecture}
\label{fig:basic-cnn}
\end{figure}

A typical architecture of one neuron can be seen in the Figure \ref{fig:basic-neuron}. That specific neuron performs an operation $ f(net) = y $ where $net = x1*w1 + x2*w2 + x3*w3 + bias$.  Bias is usually modelled as an additional input with the corresponding weight $w=1$. Then a neuron performs operation $ f(\pmb x \times \pmb w) = y$ where $\times$ denotes cross-product in between two vectors. Function $f$ is called an \textit{activation function} and usually is the same for all the neurons in the same layer. In the input layer that's just mapping from input to output, i.e. $f(x) = y$, but in the other layers, those are usually non-linear functions. 

\begin{figure}
\begin{tikzpicture}[
init/.style={
  draw,
  circle,
  inner sep=2pt,
  font=\Huge,
  join = by -latex
},
squa/.style={
  draw,
  inner sep=2pt,
  font=\Large,
  join = by -latex
},
start chain=2,node distance=13mm
]
\node[on chain=2] 
  (x2) {$x_2$};
\node[on chain=2,join=by o-latex] 
  {$w_2$};
\node[on chain=2,init] (sigma) 
  {$\displaystyle\Sigma$};
\node[on chain=2,squa,label=above:{\parbox{2cm}{\centering Activation \\ function}}]   
  {$f$};
\node[on chain=2,label=above:Output,join=by -latex] 
  {$y$};
\begin{scope}[start chain=1]
\node[on chain=1] at (0,1.5cm) 
  (x1) {$x_1$};
\node[on chain=1,join=by o-latex] 
  (w1) {$w_1$};
\end{scope}
\begin{scope}[start chain=3]
\node[on chain=3] at (0,-1.5cm) 
  (x3) {$x_3$};
\node[on chain=3,label=below:Weights,join=by o-latex] 
  (w3) {$w_3$};
\end{scope}
\node[label=above:\parbox{2cm}{\centering Bias \\ $b$}] at (sigma|-w1) (b) {};

\draw[-latex] (w1) -- (sigma);
\draw[-latex] (w3) -- (sigma);
\draw[o-latex] (b) -- (sigma);

\draw[decorate,decoration={brace,mirror}] (x1.north west) -- node[left=10pt] {Inputs} (x3.south west);
\end{tikzpicture}
\caption{A single processing unit in a neural network}
\label{fig:basic-neuron}
\end{figure}