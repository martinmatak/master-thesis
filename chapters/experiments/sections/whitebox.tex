When it comes to whitebox attacks, three approaches are evaluated: FGSM, CW and JSMA. Hyperparameters used in the FGSM and the CW attack are listed in Table \ref{table:fgsm-params} and Table \ref{table:cw-params}, respectively. In Table \ref{table:whitebox-results} results are presented.

It is interesting to notice that the FGSM attack managed to change MAE for the model with id 2 from 9.5 to 47.79. In other words, the targeted model on average predicted that a person in the adversarial image is 38 years younger or older than a person in the original image!

TODO: Add sample for FGSM fool

The CW attack, given the hyperparameters in Table \ref{table:cw-params}, managed to move MAE for model with id 1 from 6.69 to 29.45.  In other words, the targeted model on average predicted that a person in the adversarial image is 23 years younger or older than a person in the original image. This is not that much as FGSM managed for the model with id 2, but it is significant.

TODO: Add sample image for CW

Since CW attack wasn't able to find any adversarial sample for the model with id 2 with the given hyperparameters in Table \ref{table:cw-params}, further experiments are performed against that model with different values for learning rate and maximum number of iterations.

Results of those experiments are presented in Table \ref{table:cw-results}. The results show that it is not easy for CW attack to find any adversarial sample on a specific model. Although no adversarial sample is found, I performed no further exploration of combination of learning rate and number of iterations. Further argumentation of this decision can be found in Section \ref{sec:threats-to-validity}.

It is also interesting to notice that for models with id 1 and 2, the attacks find stronger adversarial samples (i.e. bigger change in MAE) than for models with id 3 and 4. Could it be that bigger models \footnote{ResNet50 architecture that is used for model 1 and 2 has 25,636,712 parameters and InceptionResNetV2  architecture that is used for model 3 and 4 has 55,873,736 parameters.} are more resistent to adversarial samples? 

As expected from the analysis of the paper \cite{DBLP:journals/corr/CarliniW16a} in Section \ref{sec:CW}, JSMA attack failed due to memory complexity of the algorithm.

\begin{table}[]
\centering
\begin{tabular}{|c|c|}
\hline
eps & 5 \\  \hline
clip min & 0  \\ \hline
clip max & 255 \\ \hline
y\_target & 10 or 90 \\ \hline
\end{tabular}
\caption{Values of the hyperparameters used in the FGSM attack}
\label{table:fgsm-params}
\end{table}

\begin{table}[]
\centering
\begin{tabular}{|c|c|}
\hline
binary\_search\_steps & 8 \\ \hline
y\_target & 10 or 90 \\ \hline
abort\_early & True \\ \hline
max\_iterations & 5000 \\ \hline
learning\_rate & 1 \\ \hline
clip\_max & 255 \\ \hline
clip\_min & 0 \\ \hline
initial\_const & 0.1 \\ \hline
\end{tabular}
\caption{Values of the hyperparameters  used in the CW attack}
\label{table:cw-params}
\end{table}

\begin{table}[]
\centering
\begin{tabular}{|cccccccc|}
\hline
\multicolumn{2}{|c|}{} & \multicolumn{2}{c|}{FGSM} & \multicolumn{2}{c|}{CW} & \multicolumn{2}{c|}{JSMA} \\ \hline
\multicolumn{1}{|c|}{\begin{tabular}[c]{@{}c@{}}model\\ id\end{tabular}} & \multicolumn{1}{c|}{\begin{tabular}[c]{@{}c@{}}clean\\ MAE\end{tabular}} & \multicolumn{1}{c|}{\begin{tabular}[c]{@{}c@{}}adv\\ MAE\end{tabular}} & \multicolumn{1}{c|}{\begin{tabular}[c]{@{}c@{}}avg\\ L2\end{tabular}} & \multicolumn{1}{c|}{\begin{tabular}[c]{@{}c@{}}adv\\ MAE\end{tabular}} & \multicolumn{1}{c|}{\begin{tabular}[c]{@{}c@{}}avg\\ L2\end{tabular}} & \multicolumn{1}{c|}{\begin{tabular}[c]{@{}c@{}}adv\\ MAE\end{tabular}} & \multicolumn{1}{c|}{\begin{tabular}[c]{@{}c@{}}avg\\ L2\end{tabular}} \\ \hline
1 & 6.69 & 32.25 & 1879.63 & 29.45 & 2952.22 & - & - \\ \hline
2 & 9.5 & 47.79 & 1879.49 & 9.5 & 0.0 & - & - \\ \hline
3 & 5.4 & 19.32 & 2510.92 & 6.39 & 1175.80 & - & - \\ \hline
4 & 4.62 & 15.03 & 2511.10 & 7.14 & 1084.08 & - & - \\ \hline
\end{tabular}
\caption{Results of different adversarial attacks. The "-" sign means that an attack couldn't be executed.}
\label{table:whitebox-results}
\end{table}

\begin{table}[]
\begin{tabular}{|c|c|c|c|c|}
\hline
max iterations & learning rate & clean MAE & adv MAE & avg L2 \\ \hline
5 000 & 1 & 9.5 & 9.5 & 0.0 \\ \hline
10 000 & 1 & 9.5 & TODO & TODO \\ \hline
10 000 & 0.1 & 9.5 & 9.5 & 0.0 \\ \hline
10 000 & 10.0 & 9.5 & 9.5 & 0.0 \\ \hline
100 000 & 0.01 & 9.5 & 9.5 & 0.0 \\ \hline
\end{tabular}
\caption{Results using the CW attack with different values of hyperparameters against model with id 2}
\label{table:cw-results}
\end{table}