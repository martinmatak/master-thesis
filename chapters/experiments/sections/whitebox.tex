When it comes to whitebox attacks, three approaches are evaluated: FGSM, CW and JSMA. In Table \ref{table:whitebox-results} results are presented. Hyperparameters used in the FGSM and the CW attack are listed in Table \ref{table:fgsm-params} and Table \ref{table:cw-params}, respectively. The results are discussed in Section \ref{sec:discussion}. 

TODO: Add sample images

\begin{table}[]
\centering
\begin{tabular}{|c|c|}
\hline
eps & 5 \\  \hline
clip min & 0  \\ \hline
clip max & 255 \\ \hline
y\_target & 10 or 90 \\ \hline
\end{tabular}
\caption{Values of the hyperparameters used in the FGSM attack}
\label{table:fgsm-params}
\end{table}

\begin{table}[]
\centering
\begin{tabular}{|c|c|}
\hline
binary\_search\_steps & 8 \\ \hline
y\_target & 10 or 90 \\ \hline
abort\_early & True \\ \hline
max\_iterations & 5000 \\ \hline
learning\_rate & 1 \\ \hline
clip\_max & 255 \\ \hline
clip\_min & 0 \\ \hline
initial\_const & 0.1 \\ \hline
\end{tabular}
\caption{Values of the hyperparameters  used in the CW attack}
\label{table:cw-params}
\end{table}

\begin{table}[]
\centering
\begin{tabular}{|cccccccc|}
\hline
\multicolumn{2}{|c|}{} & \multicolumn{2}{c|}{FGSM} & \multicolumn{2}{c|}{CW} & \multicolumn{2}{c|}{JSMA} \\ \hline
\multicolumn{1}{|c|}{\begin{tabular}[c]{@{}c@{}}model\\ id\end{tabular}} & \multicolumn{1}{c|}{\begin{tabular}[c]{@{}c@{}}clean\\ MAE\end{tabular}} & \multicolumn{1}{c|}{\begin{tabular}[c]{@{}c@{}}adv\\ MAE\end{tabular}} & \multicolumn{1}{c|}{\begin{tabular}[c]{@{}c@{}}avg\\ L2\end{tabular}} & \multicolumn{1}{c|}{\begin{tabular}[c]{@{}c@{}}adv\\ MAE\end{tabular}} & \multicolumn{1}{c|}{\begin{tabular}[c]{@{}c@{}}avg\\ L2\end{tabular}} & \multicolumn{1}{c|}{\begin{tabular}[c]{@{}c@{}}adv\\ MAE\end{tabular}} & \multicolumn{1}{c|}{\begin{tabular}[c]{@{}c@{}}avg\\ L2\end{tabular}} \\ \hline
1 & 6.21 & 29.89 & 1879.63 & 19.73 & 195.49 & - & - \\ \hline
2 & 8.61 & 41.48 & 1879.49 & 8.61 & 0.0 & - & - \\ \hline
3 & 4.82 & 23.96 & 2511.15 & TODO & 0.0 & - & - \\ \hline
4 & 4.46 & 13.67 & 2511.10 & 1TODO & 0.0 & - & - \\ \hline
\end{tabular}
\caption{Results of different adversarial attacks. The "-" sign means that an attack couldn't be executed.}
\label{table:whitebox-results}
\end{table}