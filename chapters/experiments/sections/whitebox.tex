When it comes to whitebox attacks, three approaches are evaluated: FGSM, CW and JSMA.  FGSM is used with $eps=5$ (the parameter that sets how large the perturbation may be). Number of iterations in CW is bounded to 1000. In Table \ref{table:whitebox-results} results are presented. The experiment for the CW attack with 10000 iterations yielded very similar results as with 1000 iterations. The results are discussed in the next chapter.\textbf{TODO: reference to the next chapter}.

\begin{table}[]
\begin{tabular}{|cccccccc|}
\hline
\multicolumn{2}{|c|}{} & \multicolumn{2}{c|}{FGSM} & \multicolumn{2}{c|}{CW} & \multicolumn{2}{c|}{JSMA} \\ \hline
\multicolumn{1}{|c|}{\begin{tabular}[c]{@{}c@{}}model\\ id\end{tabular}} & \multicolumn{1}{c|}{\begin{tabular}[c]{@{}c@{}}clean\\ MAE\end{tabular}} & \multicolumn{1}{c|}{\begin{tabular}[c]{@{}c@{}}adv\\ MAE\end{tabular}} & \multicolumn{1}{c|}{\begin{tabular}[c]{@{}c@{}}avg\\ L2\end{tabular}} & \multicolumn{1}{c|}{\begin{tabular}[c]{@{}c@{}}adv\\ MAE\end{tabular}} & \multicolumn{1}{c|}{\begin{tabular}[c]{@{}c@{}}avg\\ L2\end{tabular}} & \multicolumn{1}{c|}{\begin{tabular}[c]{@{}c@{}}adv\\ MAE\end{tabular}} & \multicolumn{1}{c|}{\begin{tabular}[c]{@{}c@{}}avg\\ L2\end{tabular}} \\ \hline
1 & 6.21 & 29.89 & 1879.63 & 19.73 & 195.49 & - & - \\ \hline
2 & 8.61 & 41.48 & 1879.49 & 8.61 & 0.0 & - & - \\ \hline
3 & TODO & TODO & TODO & TODO & 0.0 & - & - \\ \hline
4 & TODO& TODO & TODO & 1TODO & 0.0 & - & - \\ \hline
\end{tabular}
\caption{Results of different adversarial attacks. The "-" sign means that an attack couldn't be executed.}
\label{table:whitebox-results}
\end{table}