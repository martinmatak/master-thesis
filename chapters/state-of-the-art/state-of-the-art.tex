Machine learning is a field which is evolving quickly and therefore a lot of papers have been published in the last few years. However, since focus of this master thesis is on generating adversarial examples, related work can be separated into two topics, depending if a DNN is treated as a white-box or a black-box. 

In terms of white-box attacks, in \cite{fgsm-original}, the \textit{Fast Gradient Sign Method} is presented. It computes an adversarial image for a non-targeted attack based on the direction of the gradient of a DNN. It is presented in the Section \ref{sec:FGSM}.

In \cite{DBLP:journals/corr/PapernotMJFCS15}, the \textit{Jacobian-based Saliency Map Attack} JSMA algorithm for generating adversarial examples is presented. It is based on identifying regions in an image which have higher importance for a DNN during the classification. It is presented in the Section \ref{sec:JSMA}.

In \cite{DBLP:journals/corr/CarliniW16a}, the CW attack is presented which is based on formulating the attack as an optimization problem and using a state-of-the-art optimizer to solve it.  It is presented in the Section \ref{sec:CW}.

All three attacks, FGSM, JSMA and CW are used in the experiments in this thesis.

On the black-box side of the attacks, there is \textit{transfer-based} approach 
\cite{DBLP:journals/corr/PapernotMGJCS16}. It uses a subsitute DNN which is trained on a similar dataset as the targeted DNN. This approach is described in the Section \ref{sec:transfer-based}.

In \cite{ensemble-attack}, the authors show that adversarial samples for targeted misclassification don't transfer as well as in a pure misclassification attack. Authors suggest \textit{ensemble} approach which is described in the Section \ref{sec:ensemble-approach}.

In \cite{brendel2018decisionbased}, authors implement a completely different attack and call it \textit{Boundary Attack}. The attack starts with an image of a targeted class and then, step by step, it changes it to an image of some other class while staying adversarial, i.e. classified as a target class by a DNN under the attack. The attack is described in the Section \ref{sec:boundary-attack}.

I direct the interested reader to this survey for a detailed description \cite{survey} of the different attack strategies and defenses.

\section{Fast Gradient Sign Method (FGSM)}
\label{sec:FGSM}
Let $\pmb \theta$ be the parameters of a model, $\pmb x$ the input to the model, $y$ the target associated with $\pmb x$, and $J (\pmb \theta, \pmb x, y)$ be the cost fucntion used to train the neural network. Then an adversarial perturbation is computed as 
\[ 
\pmb \rho = \epsilon * sign (\nabla_{\pmb x} J(\pmb \theta, \pmb x, y)).
\]

An adversarial example can be crafted then by adding the adversarial perturbation to the original input

\[\pmb x = \pmb x + \pmb \rho .\]
 
 
The FGSM attack perturbs an image to increase the loss of the classifier on the resulting image. The target label in the original paper \cite{fgsm-original} was always a label with the least probability for an unmodified image. The authors evaluate their method on the ImageNet dataset \cite{datasetImageNet}, a dataset used for a large-image recognition task with 1000 classes, and achieve good results for misclassification. Targeted misclassification was not evaluated. Similar results are achieved on the MNIST dataset 	\cite{datasetMNIST}, a dataset used for a digit-recognition task (0-9), and on the CIFAR-10 dataset \cite{datasetCIFAR10}, a dataset used for a small-image recognition task, also with 10 classes as in MNIST. From Figure \ref{fig:gibbon}, a reader can get the intuition for the attack. For more details, please consult the original paper.

\begin{figure}[h]
\includegraphics[width=13cm]{gibbon}
\caption{Image taken from the \cite{fgsm-original}}
\label{fig:gibbon}
\end{figure}




\section{Jacobian-based Saliency Map Attack}
\label{sec:JSMA}
 \textit{Adversarial Saliency Maps} - maps which measure how much every pixel is important for an image to be classified as a specific class - are created. Based on them and the \textit{forward derivative} of the DNN, adversarial examples are crafted. The algorithm is validated against MNIST dataset \cite{datasetMNIST}, a digit-recognition task (0-9). The result is that using the JSMA algorithm, an attacker is able to craft a successful adversarial sample for every class.


 
\section{CW}
\label{sec:CW}
% L norm introduction

To quantify similarity between two images, different distance metrics can be used. Quantification of similarity can be used when comparing how much an adversarial image is different from the original input. There are three widely-used distance metrics in the literature for generating adversarial examples, all of which are $L_p$ distances. The $L_p$ distance is written $||\pmb x - \pmb x'||_p$, where the $p$-norm for purposes of this thesis can be defined as

\begin{equation}
  ||\pmb v||_p=\left\{
  \begin{array}{@{}ll@{}}
       |\{i | v_i \neq 0 \}|, & \text{if } p = 0 \\
    ( \sum_{i = 1}^{n} |v_i|^p)^{1/p}, & \text{if } p \in [1, \infty) \\
    max\{|v_1|, |v_2|, ..., |v_n|\} & \text{if } p = \infty
  \end{array}\right.
\end{equation} 

In other words, $L_0$ measures how many pixels are changed, $L_2$ measures standard euclidean distance and $L_\infty$ measures the maximum change to any of the coordinates. It is open for discussion which metric performs the best job in measuring the human perceptual of similarity, but neither of the $L_p$ metrics is optimal for that.

% end of L norm introduction

The authors \cite{DBLP:journals/corr/CarliniW16a} introduce three new attacks for the $L_0$, $L_2$, and $L_{ \infty }$ distance metrics. It is worth mentioning that their $L_0$ attack is the first published attack which can cause targeted misclassification on the ImageNet dataset. All three of them are based on optimization techniques.

In this thesis, $L_2$ is used in the attack and hence I explain it now.

The authors start by using the initial formulation of adversarial examples \cite{szegedy2013intriguing} and define the problem of finding an adversarial sample $\pmb x$ as follows:

\begin{align*}
\text{minimize } & \mathcal{D}(\pmb x, \pmb x + \pmb \delta) \\
\text{such that } & \mathcal{C} (\pmb x + \pmb \delta) = t \\
                  & \pmb x + \pmb \delta \in [0, 1]^n
\end{align*}

where $t$ is the target class, $\pmb \delta$ is perturbation added to the original input $\pmb x$, $\mathcal{C}$ is a function performed by the classifier, and $\mathcal{D}$ is either $L_0$, $L_2$ or $L_\infty$. 

Since the constraint $\mathcal{C} (\pmb x + \pmb \delta) = t$ is highly non-linear and therefore hard to solve directly for existing algorithms, the authors introduce the function $f$ such that  $\mathcal{C} (\pmb x + \pmb \delta) = t$ if and only if $f(\pmb x + \pmb \delta) \leq 0$. Now problem can be formulated as

\begin{align*}
\text{minimize  } & \mathcal{D}(\pmb x, \pmb x + \pmb \delta) \\
\text{such that }& f(\pmb x + \pmb \delta) \leq 0 \\
                  & \pmb x + \pmb \delta \in [0, 1]^n
\end{align*}

or using the alternative formulation: 

\begin{align*}
\text{minimize  }& \mathcal{D}(\pmb x, \pmb x + \pmb \delta) + c \cdot f(\pmb x+\pmb \delta) \\
\text{such that }& \pmb x + \pmb \delta \in [0, 1]^n
\end{align*}

where $c > 0$ is a suitably chosen constant. The authors in their implementation use a modified binary search to find the optimal value of $c$. 

Let $\pmb Z$ be the output of the targeted DNN second-to-last layer, the logits,  with $\pmb Z_i$ as an output for the class $i$.

The function $f$ that the authors find the most effective is: 

\begin{equation}
f(\pmb x + \pmb \delta) = max(max(\{\pmb Z(\pmb x + \pmb \delta)_i : i \neq t\}) - \pmb Z(\pmb x + \pmb \delta)_t, 0).
\label{fun:obj-fun}
\end{equation}

To ensure the modification yields a valid image, there is a constraint  $\pmb x + \pmb \delta \in [0, 1]^n$. The authors refer to this constraint as a "box constraint". The Adam \cite{DBLP:journals/corr/KingmaB14} optimizer does not support box constraints  natively and the authors modify the box constraint as follows in order to be able to use the Adam optimizer. A new variable $\pmb \omega$ is introduced and instead of optimizing over the variable $\pmb \delta$, an optimization is done over $\pmb \omega$, setting 

\[
\delta_i = \frac{1}{2}(tanh(\omega_i) + 1) - x_i.
\] 

Now the solution will automatically be valid since from $-1 \leq tanh(\omega_i) \leq 1$ it follows that $0 \leq x_i + \delta_i \leq 1$.

Finally, for $\mathcal{D} = L_2$, the attack can be formalized as follows. Given the original sample $\pmb x$ and the target class $t$, search for $\pmb \omega$ that solves \footnote{Here $\pmb 1$ represents a vector of same dimensionality as $\pmb x$ and $\pmb \omega$, it has value $1$ at every index and it shouldn't be confused with a ground-truth vector $\pmb 1_p$. }

\[
\text{minimize  }(||\pmb x - \frac{1}{2}(tanh(\pmb \omega) + \pmb 1)||_2)^2 + c \cdot  f(\frac{1}{2}(tanh(\pmb \omega) + \pmb 1))
\]

with $f$ similar to the objective function defined in \ref{fun:obj-fun}, but this time defined as

\[
f(\pmb x') = max(max\{\pmb Z(\pmb x ')_i : i \neq t\} - \pmb Z(\pmb x')_t, - \kappa)
\]

where $\kappa$ is a parameter that controls the confidence with which the misclassification occurs. The authors in their implementation set $\kappa = 0$. The adversarial example is then crafted as $\pmb x' = \pmb x + \pmb \delta$. For more details about the attack, please consider \cite{DBLP:journals/corr/CarliniW16a}.

According to the authors, this attack is often much more effective (and never worse) than all the others presented in the literature. Attacks are evaluated on the three datasets: ImageNet, MNIST and CIFAR-10. They also report that the JSMA attack, an attack introduced in Section \ref{sec:JSMA}, is not able to craft an adversarial example when the ImageNet dataset is used due to memory complexity of the algorithm, i.e. dimensions of images in ImageNet dataset are too big for JSMA attack. This implies that the JSMA attack would not work in my thesis as well if an image of a person is too big. Reported results for the CW attack are showing 100\% success against all three datasets.










\section{Transfer based approach}
\label{sec:transfer-based}
This technique is used to attack the DNN in the black-box settings. The idea is to create a \textit{substitute} DNN which should be similar to the targeted DNN. A precise definition of the similarity is omitted here because it's not well defined, but the substitute DNN should solve the same task as the targeted DNN.

Adversarial images are crafted then for a substitute DNN using a white-box approach, for instance the FGSM attack introduced in Section \ref{sec:FGSM}. Created adversarial images are used then as adversarial images for the black-box DNN as well. The main idea is that similar classifiers will have similar boundaries for a specific class and therefore the same adversarial example should be adversarial for both networks. 

The dataset on which the substitute neural network is trained should be similar to the dataset on which the targeted neural network is trained. Ideally, that would be the same dataset, but the assumption is that an attacker doesn't have access to that data. 

The attacker therefore generates a Synthetic Dataset. He or she starts generating the dataset by querying the targeted DNN with several examples and obtaining labels for them. Afterwards, he or she expands the dataset using the Jacobian-based Dataset Augmentation and trains the substitute neural network. For more details how to generate the synthetic dataset, please consult the original paper  \cite{DBLP:journals/corr/PapernotMGJCS16}.

The authors present good results for misclassification attacks against the MNIST dataset and the GTSRD dataset \cite{datasetGTSRD}. Targeted misclassification is not presented in the paper.


\section{Ensemble approach} 
\label{sec:ensemble-approach}
This approach is also based on transferability of an adversarial image, but instead of generating an adversarial image for one neural network, an attacker generates it for several of them. The underlying assumption is that if an adversarial example works as expected among several models, it will work as expected for the one more as well. 




\section{Boundary attack}
\label{boundary-attack}
This approach is also used in black-box settings and it is completely different from the attacks introduced in Sections \ref{sec:transfer-based} and \ref{sec:ensemble-approach}. The boundary attack has nothing to do with either a substitute DNN or transferability of the adversarial examples. 

The attack starts with an image of a targeted class and then, step by step, it changes it to an image of some other class while staying adversarial, i.e. classified as a target class by a DNN under the attack. In every iteration of the attack, the image is changed a little bit towards a class which will be in the image in the end, at least according to a human observer. More specifically, in every iteration of the attack, a perturbation that reduces the distance of the perturbed image (adversarial sample) towards the original input (an image of a class that is presented to a human observer) is added. For specific details how this perturbation is selected, please consult the original paper \cite{brendel2018decisionbased}.

After every change, the targeted DNN is queried to check if the image is still adversarial, i.e. classified as a target label. If not, the change is reverted. In this way, the attacker doesn't need any substitute neural network. 

However, this attack comes at cost of a large number of queries to the targeted DNN. For the targeted attack, the authors needed around $10^4$ queries to get an adversarial example. The real world systems could notice such intensive querying of their APIs and detect the attack. On top of that, the attacker needs both an image of the targeted class and an image of the class that will be presented to a human observer. That could be an obstacle when the number of classes is high because it can happen that it is not easy to find an image of a particular class.

The authors compare boundary attack with white-box CW attack, introduced in Section \ref{sec:CW}, on MNIST and CIFAR-10 dataset and produce only a bit worse results, although this attack is treating a targeted DNN as a black-box.  For more information about this approach, please consult the original paper \cite{brendel2018decisionbased}.

