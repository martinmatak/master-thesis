The authors \cite{ensemble-attack} show that using the transfer based approach, introduced in Section \ref{sec:transfer-based}, the target labels don't transfer well for the targeted misclassification attack. Transfer based approach seems good for a misclassification task, but not for a targeted misclassification task as well. Goal of the ensemble-based approach is to solve this issue.

The ensemble-based approach is also based on transferability of an adversarial image, but instead of generating an adversarial image for one substitute neural network, an attacker generates it for several of neural networks - \textit{the ensemble of the models}. The underlying assumption is that if an adversarial example works as expected among several models, there is a higher probability that it will work as expected for the one more as well. 

Formally, given $k$ white-box models with softmax outputs being $\pmb J_1$, ..., $\pmb J_k$, an original image $\pmb x$, and its ground-truth vector $\pmb y$, the ensemble-based approach solves the following optimization problem:

\begin{equation}\label{eq:ensemble-objective}
argmin_{\pmb x^*} - log ((\sum_{i=1}^k \alpha_i \pmb J_i(\pmb x^*)) \cdot \pmb y^*) + \lambda d(\pmb x, \pmb x^*)
\end{equation}

where $\pmb y^*$ is a vector encoding the target label specified by the adversary, $\sum_{i=1}^k \alpha_i \pmb J_i(\pmb x^*)$ is the ensemble model, and $\alpha_i$ are the ensemble weights s.t. $\sum_{i=1}^k \alpha_i = 1$.

The authors use two approaches to solve this problem: optimization-based and fast gradient-based. Optimization-based approach uses state-of-the-art optimizer to solve the optimization problem defined in Equation \ref{eq:ensemble-objective}. The authors observe that optimization-based approach outputs a large proportion of the targeted adversarial images whose target labels can transfer. The exact percentage depends on the architectures used for the ensemble models and for the targeted DNN and can vary from 11\% up to 99\%. Fast gradient-based approach using the ensemble model gives result no better than using the single model.