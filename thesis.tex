\documentclass[draft,final]{vutinfth} % Remove option 'final' to obtain debug information.

% Load packages to allow in- and output of non-ASCII characters.
\usepackage{lmodern}        % Use an extension of the original Computer Modern font to minimize the use of bitmapped letters.
\usepackage[T1]{fontenc}    % Determines font encoding of the output. Font packages have to be included before this line.
\usepackage[utf8]{inputenc} % Determines encoding of the input. All input files have to use UTF8 encoding.

% Extended LaTeX functionality is enables by including packages with \usepackage{...}.
\usepackage{amsmath}    % Extended typesetting of mathematical expression.
\usepackage{amssymb}    % Provides a multitude of mathematical symbols.
\usepackage{mathtools}  % Further extensions of mathematical typesetting.
\usepackage{microtype}  % Small-scale typographic enhancements.
\usepackage[inline]{enumitem} % User control over the layout of lists (itemize, enumerate, description).
\usepackage{multirow}   % Allows table elements to span several rows.
\usepackage{booktabs}   % Improves the typesettings of tables.
\usepackage{subcaption} % Allows the use of subfigures and enables their referencing.
\usepackage[ruled,linesnumbered,algochapter]{algorithm2e} % Enables the writing of pseudo code.
\usepackage[usenames,dvipsnames,table]{xcolor} % Allows the definition and use of colors. This package has to be included before tikz.
\usepackage{nag}       % Issues warnings when best practices in writing LaTeX documents are violated.
\usepackage{todonotes} % Provides tooltip-like todo notes.
\usepackage{hyperref}  % Enables cross linking in the electronic document version. This package has to be included second to last.
\usepackage[acronym,toc]{glossaries} % Enables the generation of glossaries and lists fo acronyms. This package has to be included last.

% Define convenience functions to use the author name and the thesis title in the PDF document properties.
\newcommand{\authorname}{Martin Matak} % The author name without titles.
\newcommand{\thesistitle}{Attacks on Neural Networks} % The title of the thesis. The English version should be used, if it exists.

% Set PDF document properties
\hypersetup{
    pdfpagelayout   = TwoPageRight,           % How the document is shown in PDF viewers (optional).
    linkbordercolor = {Melon},                % The color of the borders of boxes around crosslinks (optional).
    pdfauthor       = {\authorname},          % The author's name in the document properties (optional).
    pdftitle        = {\thesistitle},         % The document's title in the document properties (optional).
    pdfsubject      = {Subject},              % The document's subject in the document properties (optional).
    pdfkeywords     = {a, list, of, keywords} % The document's keywords in the document properties (optional).
}

\setpnumwidth{2.5em}        % Avoid overfull hboxes in the table of contents (see memoir manual).
\setsecnumdepth{subsection} % Enumerate subsections.

\nonzeroparskip             % Create space between paragraphs (optional).
\setlength{\parindent}{0pt} % Remove paragraph identation (optional).

\makeindex      % Use an optional index.
\makeglossaries % Use an optional glossary.
%\glstocfalse   % Remove the glossaries from the table of contents.

% Set persons with 4 arguments:
%  {title before name}{name}{title after name}{gender}
%  where both titles are optional (i.e. can be given as empty brackets {}).
\setauthor{Pretitle}{\authorname}{Posttitle}{male}
\setadvisor{Pretitle}{Georg Weissenbacher}{Posttitle}{male}

% For bachelor and master theses:
%\setfirstassistant{Pretitle}{Forename Surname}{Posttitle}{male}
%\setsecondassistant{Pretitle}{Forename Surname}{Posttitle}{male}
%\setthirdassistant{Pretitle}{Forename Surname}{Posttitle}{male}

% For dissertations:
%\setfirstreviewer{Pretitle}{Forename Surname}{Posttitle}{male}
%\setsecondreviewer{Pretitle}{Forename Surname}{Posttitle}{male}

% For dissertations at the PhD School and optionally for dissertations:
%\setsecondadvisor{Pretitle}{Forename Surname}{Posttitle}{male} % Comment to remove.

% Required data.
\setaddress{Address}
\setregnumber{0123456}
\setdate{01}{01}{2001} % Set date with 3 arguments: {day}{month}{year}.
\settitle{\thesistitle}{Titel der Arbeit} % Sets English and German version of the title (both can be English or German). If your title contains commas, enclose it with additional curvy brackets (i.e., {{your title}}) or define it as a macro as done with \thesistitle.
%\setsubtitle{Optional Subtitle of the Thesis}{Optionaler Untertitel der Arbeit} % Sets English and German version of the subtitle (both can be English or German).

% Select the thesis type: bachelor / master / doctor / phd-school.
% Bachelor:
%\setthesis{bachelor}
%
% Master:
\setthesis{master}
\setmasterdegree{dipl.} % dipl. / rer.nat. / rer.soc.oec. / master
%
% Doctor:
%\setthesis{doctor}
%\setdoctordegree{rer.soc.oec.}% rer.nat. / techn. / rer.soc.oec.
%
% Doctor at the PhD School
%\setthesis{phd-school} % Deactivate non-English title pages (see below)

% For bachelor and master:
\setcurriculum{Logic and Computation}{Logic and Computation} % Sets the English and German name of the curriculum.

% For dissertations at the PhD School:
%\setfirstreviewerdata{Affiliation, Country}
%\setsecondreviewerdata{Affiliation, Country}


\begin{document}

\frontmatter % Switches to roman numbering.
% The structure of the thesis has to conform to
%  http://www.informatik.tuwien.ac.at/dekanat

\addtitlepage{naustrian} % German title page (not for dissertations at the PhD School).
\addtitlepage{english} % English title page.
\addstatementpage

\begin{acknowledgements*}
\todo{Enter your text here.}
\end{acknowledgements*}

\begin{kurzfassung}
\todo{Ihr Text hier.}
\end{kurzfassung}

\begin{abstract}
\todo{Enter your text here.}
\end{abstract}

% Select the language of the thesis, e.g., english or naustrian.
\selectlanguage{english}

% Add a table of contents (toc).
\tableofcontents % Starred version, i.e., \tableofcontents*, removes the self-entry.

% Switch to arabic numbering and start the enumeration of chapters in the table of content.
\mainmatter

\chapter{Introduction}
This is introduction chapter.	
\section{Motivation and Problem Definition}
As Artificial Intelligence (AI) is getting a greater impact in our everyday life, it is of utmost importance that it is safe for humans.  

One of the tools used in AI are deep neural networks (DNNs) and neural networks in general. Deep neural networks are powerful learning models that achieve excellent performance on visual recognition problems \cite{krizhevsky2012imagenet}. Those results imply that DNNs can be used in different industry domains, e.g. traffic sign recognition, a domain in which DNNs even outperform humans \cite{outperformhumans}. 

Nevertheless, neural networks tend to have some peculiar properties as well. If only several pixels in an image are changed, some neural networks produce incorrect results \cite{szegedy2013intriguing}. Such a behaviour is not allowed in safety-critical systems. For example, if an autonomous car recognizes a  \textit{STOP} sign as anything else but a \textit{STOP} sign, it can lead to deathly consequences. Therefore, it is important to verify the system. 

The assumption in the rest of this proposal is that the neural network is performing a \textit{classification task} - mapping an input (image) to a discrete output value (e.g. mapping an image of a traffic sign to the name of the traffic sign).  Output values are often called labels or categories. There is a finite number of possible labels.

To find errors in a system where a neural network is used, we need a framework which can trigger every possible output of the neural network.  We want to achieve that by using only one image. 

Since for the same input we will always get the same output, modifications of that image are necessary. For a different desired output, different modification on the image is performed.
Then we repeat that process by changing our desired output in every iteration and in that way, cover all the possible outputs. Of course, idea is that the modified image is as close as possible to the original one. For example, an image of a  \textit{STOP} sign can be modified as long as it still is an image of a  \textit{STOP} sign to a human observer. In that case, with the various images of  \textit{STOP} sign we cover all the possible outputs of the neural network - all other traffic signs. Using this technique, we can trigger errors in the system which can help us in the process of its verification. 

Modifying an input for a neural network with a goal of reaching an output, which is a different than the output of an unmodified input, is an attack called \textit{misclassification}. If an output which an attacker wants to reach is one specific label, then a name of the attack is \textit{targeted misclassification}. Creating a framework for targeted misclassification in the domain of age estimation is the main focus of this thesis.

\section{Aim of the Work}
To the best of my knowledge, nobody executed an attack against a DNN in the domain of age estimation yet. Hence, one of the first questions which this thesis aims to answer is the question whether already existing techniques can be used for attacks in this domain.

In order to further define aim of this work, an explanation of the major difference in between two types of attacks against a neural network is necessary, namely the difference in between \textit{white-box} attack and \textit{black-box} attack. 
In white-box attacks, an adversary has all the information about the DNN under attack. The internal structure of the neural network, all the implementation details and values of all the variables in any moment are known to the attacker. In other words, the attacker has access to the source code and nothing is hidden.
On the other hand, in black-box attacks an adversary doesn't have access to all the information as in white-box. Depending on the precise definition of the black-box, more or less information is provided. In this thesis, the only capability of the black-box adversary is to observe a label given by the DNN to chosen inputs.

A natural question is - which algorithm is the best one for the white-box attack and which one for the black-box attack? Several algorithms will be run in the different settings and results will be compared. That will answer a question which algorithm to use in which scenario.

Furthermore, in the domain of age estimation, it makes sense to be less strict about the targeted label. For instance, if an image of a minor is classified as an image of a person over a fifty years old, this could lead to the same consequences, no matter if it is classified as a 55 or 65 years old person. More precisely, if the goal is to hit any label from a specific group of labels, I define that attack as \textit{semi-targeted misclassification}. Of course, a group of labels must be smaller than a whole set of possible labels, otherwise the task is trivial. In general, this attack makes sense in any environment where labels can somehow be ordered.

One already existing black-box approach will be adapted to this more relaxed setting. Such a relaxation is not yet introduced in the literature since it's very domain specific so results can't be compared against a previous work, but results can be and will be compared with targeted black-box misclassification attacks. This comparison is explained in more details in Section \ref{approach}. 

In terms of development, a DNN will be implemented which will receive an image as an input and output will be how old the person in the image is. In other words, a DNN for age estimation will be trained. All the attacks will be executed against this DNN. 

Next, the framework will be constructed for a white-box and a black-box targeted misclassification attack. In other words, while treating the DNN as a white-box or a black-box, images will be constructed in a way that the targeted DNN outputs a specific year. Finally, the framework will be expanded by the ability to craft semi-targeted black-box misclassification attacks.

\section{Methodological Approach} \label{approach}
The methodological approach consists of the following steps:
\begin{enumerate}
    \item Existing methods for training a neural network will be identified and explained. Goal is to give a brief introduction in this area so that non-expert reader can follow the rest of the thesis. These methods will be used to train a deep neural network for age estimation of a person in the image.
    
    \item Research about the different state of the art methods for generating \textit{adversarial examples}, i.e. images which are not correctly classified by DNN needs to be conducted. Focus here is on the approaches that have a possibility of targeted misclassification. While treating a DNN from the first step as a white-box, those methods will be used to construct adversarial examples. Results of the attacks will be presented and analyzed.
    
    \item Research about the targeted black-box attack methods will be done and those methods will explained and used to generate an adversarial inputs for a DNN  from the first step, but this time while treating it as a black box. Results of different algorithms will be compared. 
    
    \item As the last step, combining the domain knowledge and an existing attack method, a new algorithm will be implemented and images for semi-targeted misclassification will be constructed. Results will be compared against targeted black-box approaches. The way it will be compared is the following: a target label for a targeted version of the attack will be the median value of the set of labels in a semi-targeted version of the attack. If a DNN outputs any label out of the set of the labels, an attack will be considered as a successful no matter if it is a targeted or a semi-targeted misclassification attack.
\end{enumerate}

\section{State-of-the-Art}
Machine learning is a field which is evolving quickly and therefore a lot of papers have been published in the last few years. However, since focus of this master thesis is on generating adversarial examples, related work can be separated into two topics, depending if a DNN is treated as a white-box or a black-box:

In terms of white-box attacks, in \cite{fgsm-original}, the \textit{Fast Gradient Sign Method} is presented. It computes an adversarial image for a non-targeted attack based on the direction of the gradient of a DNN. In this thesis, a targeted version \cite{fgsm-targeted} of it will be used. The target label in the paper was always a label with the least probability for an unmodified image. They evaluate their method on the ImageNet dataset \cite{datasetImageNet}, a large-image recognition task with 1000 classes. Since the goal in the paper was misclassification, results for a targeted misclassification are not presented. 

In \cite{DBLP:journals/corr/PapernotMJFCS15}, the JSMA algorithm for generating adversarial examples is presented. It is based on identifying regions in an image which have higher importance for a DNN during the classification. \textit{Adversarial Saliency Maps} - maps which measure how much every pixel is important for an image to be classified as a specific class - are created. Based on them and the \textit{forward derivative} of the DNN, adversarial examples are crafted. The algorithm is validated against MNIST dataset \cite{datasetMNIST}, a digit-recognition task (0-9). The result is that using the JSMA algorithm, an attacker is able to craft a successful adversarial sample for every class.

In \cite{DBLP:journals/corr/CarliniW16a}, the CW attack is presented which is based on formulating the attack as an optimization problem and using a state-of-the-art optimizer to solve it. According to the authors, this attack is often much more effective (and never worse) than all the others presented in the literature. 
Attacks are evaluated on three datasets: ImageNet, MNIST and CIFAR-10 \cite{datasetCIFAR10}, a small-image recognition task, also with 10 classes as in MNIST. They also report that JSMA is always failing on ImageNet dataset due to memory complexity of the algorithm, i.e. ImageNet dimensions of an image (229x229x3) are too big for JSMA. This implies that JSMA could not work in my thesis as well if an image of a person is too big. Reported results for CW attack are showing 100\% success against all three datasets.

All three attacks, FGSM, JSMA and CW will be used and explained in this thesis.

On the black-box side of the attacks, there is \textit{transfer-based} approach 
\cite{DBLP:journals/corr/PapernotMGJCS16}. It uses a subsitute DNN which is trained on a similar dataset as the targeted DNN. Adversarial images are crafted then for a substitute DNN using a white-box approach. Those images are used then as adversarial images for a black-box DNN as well.
The Authors present results for misclassification attack against MNIST dataset and GTSRD dataset \cite{datasetGTSRD}. 

In \cite{ensemble-attack}, the authors show that adversarial samples for targeted misclassification don't transfer as well as in a pure misclassification attack. Authors suggest \textit{ensemble} approach. This approach is also based on transferability of an adversarial image, but instead of generating an adversarial image for one neural network, generates it for several of them. The underlying assumption is that if an adversarial example works as expected among several models, it will work as expected for the one more as well. Both approaches will be implemented - when a substitute network is only a single neural network, and when there is several of them. I want to see if it's enough to use only one neural network as a substitute to achieve good results in semi-targeted misclassification.

In \cite{brendel2018decisionbased}, authors implement a completely different attack and call it \textit{Boundary Attack}. The attack starts with an image of a targeted class and then, step by step, it changes it to an image of some other class while staying adversarial, i.e. classified as a target class by a DNN under the attack. In every iteration of the attack, the image is changed a bit towards a class which will be in the image in the end, according to a human observer. After every change, the DNN is queried to check if the image is still adversarial, i.e. classified as a target label. If not, the change is reverted. In this way, they don't need any substitute neural network, but they do need a lot of queries to the targeted DNN. Authors compare this attack with CW attack on MNIST and CIFAR-10 and produce only a bit worse results, although this attack is treating a DNN as a black-box. This approach will be compared with the \textit{ensemble} approach in this thesis.

I direct the interested reader to this survey for a detailed description \cite{survey} of the different attack strategies and defenses.


\section{Relevance to Logic and Computation}
This thesis has a tight correspondence to the topics of the module \textit{Knowledge Representation and Artificial Intelligence} of the master's program \textit{Logic and Computation}. Some of these topics include machine learning, neural networks and artificial intelligence in general.

The following courses are related to the topics of the thesis:
\begin{itemize}
    \item 184.702 Machine Learning
    \item 188.501 Similarity Modeling 1
    \item 188.498 Similarity Modeling 2
    \item 107.386 Classification and Discriminant Analysis
    \item 188.962 Seminar in Artificial Intelligence
\end{itemize}



\chapter{Additional Chapter}
\todo{Enter your text here.}


\backmatter

% Use an optional list of figures.
%\listoffigures % Starred version, i.e., \listoffigures*, removes the toc entry.

% Use an optional list of tables.
%\cleardoublepage % Start list of tables on the next empty right hand page.
%\listoftables % Starred version, i.e., \listoftables*, removes the toc entry.

% Use an optional list of alogrithms.
%\listofalgorithms
%\addcontentsline{toc}{chapter}{List of Algorithms}

% Add an index.
\printindex

% Add a glossary.
\printglossaries

% Add a bibliography.
\bibliographystyle{alpha}
\bibliography{thesis-bibliography}

\end{document}